

\documentclass[12pt,chapterheads]{ucsd}

%% AMS PACKAGES
\usepackage{amsmath, amscd, amssymb, amsthm}
\usepackage{scrextend}
\usepackage{pslatex}
\usepackage{graphicx}

%% CAPTION
\makeatletter
\gdef\@ptsize{2}% 12pt documents
\let\@currsize\normalsize
\makeatother
\usepackage{setspace}
\doublespace
\usepackage[font=small, width=0.9\textwidth]{caption}

%% SUBFIG - Use this to place multiple images in a
%    single figure.  Subfig will handle placement and
%    proper captioning (e.g. Figure 1.2(a))
% \usepackage{subfig}

%% TIMES FONT - replacements for Computer Modern
%%   This package will replace the default font with a
%%   Times-Roman font with math support.
% \usepackage[T1]{fontenc}
% \usepackage{mathptmx}

%% INDEX
%   Uncomment the following two lines to create an index: 
% \usepackage{makeidx}
% \makeindex
%   You will need to uncomment the \printindex line near the
%   bibliography to display the index.  Use the command
% \index{keyword} 
%   within the text to create an entry in the index for keyword.
%   To compile a LaTeX document with an index the 'makeindex'
%   command will need to be run.  See the wiki for more details.

%% HYPERLINKS
%   To create a PDF with hyperlinks, you need to include the hyperref package.
%   THIS HAS TO BE THE LAST PACKAGE INCLUDED!
%   Note that the options plainpages=false and pdfpagelabels exist
%   to fix indexing associated with having both (ii) and (2) as pages.
%   Also, all links must be black according to OGS.
%   See: http://www.tex.ac.uk/cgi-bin/texfaq2html?label=hyperdupdest
%   Note: This may not work correctly with all DVI viewers (i.e. Yap breaks).
%   NOTE: hyperref will NOT work in draft mode, as noted above.
% \usepackage[colorlinks=true, pdfstartview=FitV, 
%             linkcolor=black, citecolor=black, 
%             urlcolor=black, plainpages=false,
%             pdfpagelabels]{hyperref}
% \hypersetup{ pdfauthor = {Your Name Here}, 
%              pdftitle = {The Title of The Dissertation}, 
%              pdfkeywords = {Keywords for Searching}, 
%              pdfcreator = {pdfLaTeX with hyperref package}, 
%              pdfproducer = {pdfLaTeX} }
% \urlstyle{same}
% \usepackage{bookmark}


%% CITATIONS
% Sets citation format
% and fixes up citations madness
\usepackage{microtype}  % avoids citations that hang into the margin


%% FOOTNOTE-MAGIC
% Enables footnotes in tables, re-referencing the same footnote multiple times.
\usepackage{footnote}
\makesavenoteenv{tabular}
\makesavenoteenv{table}


%% TABLE FORMATTING MADNESS
% Enable all sorts of fun table tricks
\usepackage{rotating}  % Enables the sideways environment (NCPW)
\usepackage{array}  % Enables "m" tabular environment http://ctan.org/pkg/array
\usepackage{booktabs}  % Enables \toprule  http://ctan.org/pkg/array



\begin{document}

%% FRONT MATTER
%
%  All of the front matter.
%  This includes the title, degree, dedication, vita, abstract, etc..
%  Modify the file template_frontmatter.tex to change these pages.
\include{frontmatter}





%% DISSERTATION

% A common strategy here is to include files for each of the chapters. I.e.,
% Place the chapters is separate files: 
%   chapter1.tex, chapter2.tex
% Then use the commands:
%   \include{chapter1}
%   \include{chapter2}
%
% Of course, if you prefer, you can just start with
%   \chapter{My First Chapter Name}
% and start typing away.  
\chapter{Just a Test}
This is only a test
\section{A section}
Lorem ipsum dolor sit amet, consectetuer adipiscing elit. Nulla odio
sem, bibendum ut, aliquam ac, facilisis id, tellus. Nam posuere pede
sit amet ipsum. Etiam dolor. In sodales eros quis pede.  Quisque sed
nulla et ligula vulputate lacinia. In venenatis, ligula id semper
feugiat, ligula odio adipiscing libero, eget mollis nunc erat id orci.
Nullam ante dolor, rutrum eget, vestibulum euismod, pulvinar at, nibh.
In sapien. Quisque ut arcu. Suspendisse potenti. Cras consequat cursus
nulla.

\subsection{A Figure Example}
\label{ssec:figure_example}

This subsection shows a sample figure.

\begin{figure}[h] 
  \centering
  \includegraphics[width=0.5\textwidth]{sandiego}
  \caption[Short figure caption (must be \protect{$< 4$} lines in the list of figures)]{A picture of San Diego.  Note that figures must be on their own line (no neighboring text) and captions must be single-spaced and appear \protect\textit{below} the figure.  Captions can be as long as you want, but if they are longer than 4 lines in the list of figures, you must provide a short figure caption.\index{SanDiego}} 
  \label{fig:sandiego}
\end{figure}

\subsection{A Table Example}

While in Section \ref{ssec:figure_example} Figure \ref{fig:sandiego} we had a majestic figure, here we provide a crazy table example.


%%%% TABLE 1 %%%%
\vspace{0.25in}
\begin{table}[!ht]
\caption[Short figure caption (must be \protect{$< 4$} lines in the list of tables)]{A table of when I get hungry.  Note that tables must be on their own line (no neighboring text) and captions must be single-spaced and appear \protect\textit{above} the table.  Captions can be as long as you want, but if they are longer than 4 lines in the list of figures, you must provide a short figure caption.}

\vspace{-0.25in}
\begin{center}
\begin{tabular}{|p{1in}|p{2in}|p{3in}|}

\hline
Time of day & Hunger Level & Preferred Food \\

\hline
8am & high & IHOP (French Toast) \\

\hline
noon & medium & Croutons (Tomato Basil Soup \& Granny Smith Chicken Salad) \\

\hline
5pm & high & Bombay Coast (Saag Paneer) or Hi Thai (Pad See Ew) \\

\hline
8pm & medium & Yogurt World (froyo!) \\

\hline
\end{tabular}
\end{center}
\label{tab:analysis3}
\end{table}



%% APPENDIX
\appendix
\chapter{Final notes}
What to do about things \cite{Martin_1983}.  What did he say \cite{Rilling_Insel_1999}.
  Remove me in case of abdominal pain.



%% END MATTER
% \printindex %% Uncomment to display the index
% \nocite{}  %% Put any references that you want to include in the bib 
%               but haven't cited in the braces.
\bibliographystyle{alpha}  %% This is just my personal favorite style. 
%                              There are many others.
%\setlength{\bibleftmargin}{0.25in}  % indent each item
%\setlength{\bibindent}{-\bibleftmargin}  % unindent the first line
%\def\baselinestretch{1.0}  % force single spacing
%\setlength{\bibitemsep}{0.16in}  % add extra space between items
\bibliography{references}  %% This looks for the bibliography in template.bib 
%                          which should be formatted as a bibtex file.
%                          and needs to be separately compiled into a bbl file.
\end{document}

